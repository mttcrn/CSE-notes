\documentclass{article}
\usepackage{geometry}
\geometry{a4paper, top=3cm, bottom=3cm, left=3.5cm, right=3.5cm, heightrounded, bindingoffset=5mm}

\usepackage{float}
\usepackage{graphicx}
\graphicspath{{./images/}}

\usepackage{amssymb}
\usepackage{amsmath}
\usepackage{amsthm}

\theoremstyle{definition}
\newtheorem{definizione}{Definizione}

\theoremstyle{plain}
\newtheorem{proposizione}{Propozisione}

\theoremstyle{plain}
\newtheorem{corollario}{Corollario}

\theoremstyle{plain}
\newtheorem{teorema}{Teorema}

\theoremstyle{plain}
\newtheorem{esercizio}{Esercizio}

\usepackage{hyperref}

\title{Logica e Algebra 2 - argomenti di esame}
\author{Caterina Motti}
\date{Sessione invernale 2024}

\begin{document}

\maketitle
\tableofcontents
\clearpage

\section{Campi finiti}
\subsection{Prerequisiti sui monoidi, gruppi e anelli.}
\begin{definizione}
    Un \textbf{monoide} è un insieme X con un'operazione associativa e un'identità.
\end{definizione}
\begin{definizione}
    Sia X un monoide. Un elemento $x \in X$ è \textbf{invertibile} se $\exists y \in X$ tale che $xy = yx = e$. Se $x \in X$ è invertibile, si indica con $x^{-1}$ il suo inverso ed è unico. \newline
    L'identità del monoide è sempre invertibile e il suo inverso è l'identità stessa.
\end{definizione}
NB: l'insieme degli elementi invertibili di $(\mathbb{Z}, +)$ è $\mathbb{Z}$.
\begin{definizione}
    Un \textbf{gruppo} è un monoide in cui tutti gli elementi sono invertibili. Se l'operazione è commutativa si dice abeliano o commutativo.
\end{definizione}
NB: $(\mathbb{Z}, +)$ è un gruppo abeliano.
\begin{definizione}
    Sia $X$ un gruppo con identità "$e$". Un sottoinsieme $Y \subseteq X$ tale che $e \in Y$ e con l'operazione indotta è un gruppo, si dice \textbf{sottogruppo}. \newline
    Ogni gruppo contiene almeno due sottogruppi: il gruppo stesso e il sottogruppo banale $\{e\}$. \newline
    Un sottogruppo di un gruppo finito, abeliano o ciclico mantiene la rispettiva proprietà. 
    Inoltre, se $G$ è un gruppo finito, e $H \subseteq G$ è un sottogruppo allora: $[g] \in G/H \implies |[g]| = |H|$. Poichè le classi di equivalenza sono una partizione di $G$ allora $|G| = |G/H|\cdot|H|$. Quindi la cardinalità (o ordine) di un sottogruppo di un gruppo finito divide la cardinalità del gruppo stesso.
\end{definizione}
\begin{definizione}
    Un insieme X su cui sono definite due operazioni ("+", "$\cdot$") è detto \textbf{anello}. Inoltre:
    \begin{itemize}
        \item $(X, +)$ è un gruppo abeliano con identità $0$.
        \item $(X, \cdot)$ è un monoide con identità $1_A$ detta unità dell'anello.
        \item vale la proprietà distributiva su "+" e "$\cdot$".
    \end{itemize}
    Si dice che l'anello è commutativo se il monoide $(X, \cdot)$ è commutativo.
    Considero sempre $0 \neq 1_A$, e con "anello" s'intende sempre "anello con unità".
\end{definizione}

\subsection{Sottogruppi di $(\mathbb{Z}, +)$.}
\begin{teorema}\label{teo: sott_z}
    L'insieme dei sottogruppi di $(\mathbb{Z}, +)$ è $\{n\mathbb{Z} : n \in \mathbb{N}\}$,
\end{teorema}
\begin{proof}[Dimostrazione]
    Sia $H \subseteq \mathbb{Z}$.
    Se $H = \{0\}$ allora $H = 0\mathbb{Z} = \emptyset$ (analogo con $n=1$). \newline
    Sia H un sottogruppo non banale. 
    Sia $H_{>0} := \{h \in H : h > 0\}$. Definisco $k:= min(H_{>0})$, allora preso $h \in H_{>0}$ con $h \neq k$ si ha che $h > k$ e quindi posso scriverlo come $h = nk + r$, $n \in \mathbb{N}$, $0 \le r < k$.
    Dunque $r = h-nk$ a sua volta $\in H \implies r = 0$ per minimalità di k infatti $ r < k $. Per cui $h \in n \mathbb{Z}$
\end{proof}

\subsection{Ogni sottogruppo di un gruppo ciclico è ciclico.}
\begin{definizione}
    Un gruppo G è \textbf{ciclico} se $G = <g>$ per qualche $g \in G$. Si dice che g genera G.
\end{definizione}
Nota: un gruppo ciclico è abeliano.

\begin{teorema}\label{teo:sott_cicl}
    Sia G un gruppo ciclico (quindi generato da un solo elemento), allora ogni sottogruppo di G è ciclico. 
\end{teorema}
\begin{proof}[Dimostrazione]
    Sia $g \in G$ tale che $G = <g>$.
    La funzione $\varphi : (\mathbb{Z}, +) \rightarrow G$ definita da $\varphi (n) = g^n$ $\forall n \in \mathbb{Z}$ è un morfismo di gruppi infatti $\varphi(a+b) = g^{a+b} = g^a \cdot g^b = \varphi(a) \cdot \varphi(b)$ ed è suriettivo in quanto $G = <g>$.
    \begin{enumerate}
        \item G è infinito: allora $Ker(\varphi) = \{0\}$ e quindi $\varphi$ è iniettivo in quanto contiene l'elemento neutro del gruppo. Dunque $\varphi$ è un isomorfismo di gruppi. Poichè tutti i sottogruppi di $\mathbb{Z}$ sono ciclici allora anche quelli di G lo sono.\\
        Se per assurdo $x \in Ker(\varphi)$ e $x \neq 0$ allora si avrebbe che $g^x = e \implies g^{-x} = e \implies |G| \le 2x + 1$, che è assurdo in quanto il gruppo è infinito e ciclico (ci sarebbero due rappresentanti uguali). 
        \item G è finito: sia $H \subseteq G$ un sottogruppo. Allora $\varphi^{-1}(H) := \{n \in \mathbb{Z} : \varphi(n) \in H\} \subseteq \mathbb{Z}$, quindi dato che tutti i sottogruppi di $\mathbb{Z}$ sono ciclici (teorema \ref{teo: sott_z}) esiste $k \in \mathbb{N}$ tale che $\varphi^{-1}(H) = <k>$. \newline
        La restrizione $\varphi : k\mathbb{Z} \rightarrow H$ è un morfismo suriettivo di gruppi, infatti $\varphi(hk) = \varphi(\underbrace{k + .. + k}_{h\ volte} ) = \underbrace{\varphi(k) \cdot \varphi(k)}_{h\ volte} = [\varphi(k)]^h$ $\forall h, k \in \mathbb{Z}$. Quindi $H = <\varphi(k)>$.
    \end{enumerate}
\end{proof}
\begin{corollario}
    L'insieme dei sottogruppi di $\mathbb{Z}_n$ con $n \in \mathbb{N}$ è $\{<\overline{m}> : \overline{m} \in \mathbb{Z}_n\}$
\end{corollario}
\subsection{Teorema di isomorfismo per gruppi abeliani e per anelli commutativi.}
\subsubsection*{Gruppi abeliani}
\begin{teorema}\label{teo:iso}
    Sia $f: G_1 \rightarrow G_2$ un morfismo di gruppi. Allora esiste un morfismo iniettivo $\Psi : G_1 / {Ker(f)} \rightarrow G_2$ tale che il seguente diagramma è commutativo.

    \begin{figure}[h]
        \centering
        \includegraphics[scale = 0.6]{images/iso-gruppi.png}
        \caption{Isormorfismo di gruppi}
        \label{fig:iso-gruppi}
    \end{figure}
    
    In particolare, $G_1 / {Ker(F)} \simeq Im(f)$ e $f = \Psi \cdot \pi$.
\end{teorema}
\begin{proof}[Dimostrazione]
    L'assegnazione $[g] \mapsto f(g)$, $\forall g \in G_1$, definisce una funzione $\Psi : G_1 / {Ker(f)} \rightarrow G_2$.
    Infatti, se $g' \sim g$ ovvero $[g] = [g']$, allora $g = g' + h$ con $h \in Ker(f)$. Dunque $f(g) = f(g' + h) = f(g') + f(h) = f(g')$. Dunque $\Psi$ è ben definita. \\
    Poichè $f$ è morfismo di gruppi anche $\Psi$ lo è infatti: $\Psi([g] + [g']) = \Psi([g + g']) = f(g + g') = f(g) + f(g') = \Psi([g]) + \Psi[g']$. \\
    $\Psi$ è iniettivo infatti: $Ker(f) = \{[g] \in G_1/Ker(f) : \Psi([g]) = 0_2\} = \{[g] \in G_1/ker(f) : f(g) = 0_2\} = \{[0_1]\}$.\\
    $\Psi$ è suriettivo infatti: $\forall x \in Im(f)$ $\exists g \in G$ tale che $x = f(g) = \Psi([g])$. \\
    Infine, $\Psi : G_1 / {Ker(f)} \rightarrow Im(f)$ è un morfismo di gruppi, iniettivo e suriettivo, quindi isomorfo.
\end{proof}

\subsubsection*{Anelli commutativi}
\begin{teorema}
    Sia $f:A_1 \rightarrow A_2$ un morfismo di anelli commutativi. Allora esiste un morfismo iniettivo di anelli $\Psi : A_1 / {Ker(f)} \rightarrow A_2$ tale che il seguente diagramma è commutativo.

    \begin{figure}[h]
        \centering
        \includegraphics[scale = 0.6]{images/iso-anelli.png}
        \caption{Isormorfismo di anelli}
        \label{fig:iso-anelli}
    \end{figure}

\end{teorema}
\begin{proof}[Dimostrazione]
    Partendo dal teorema di isormorfimo per gruppi abeliani (teorema \ref{teo:iso}) basta dimostrare che $\Psi$ è un morfismo di anelli. In particolare, $\Psi([x][y]) = \Psi([x \cdot y]) = f(x \cdot y) = f(x)f(y) = \Psi([x])\Psi([y])$ $\forall x, y \in A_1$ (oltre all'operazione '$+$', anche '$\cdot$ è associativa).
\end{proof}

\subsection{Teorema cinese dei resti.}
Notazione: sia $\overline{x} \in \mathbb{Z}_n$. La classe di equivalenza  $\overline{x}$ è scritta anche come $x \mod n$.

\begin{teorema}\label{teo:TCR}
    Siano $n_1, n_2, .., n_k \in \mathbb{N}\backslash\{0, 1\}$ tali che $MDC\{n_i, n_j\} = 1$ $\forall 1 \le i, j \le k$ e $i \neq j$. Sia $n:= n_1 \cdot .. \cdot n_k$ allora la funzione $\Psi : \mathbb{Z}_n \rightarrow \mathbb{Z}_{n_1} \times .. \times \mathbb{Z}_{n_k}$ è un isomorfismo di anelli.
\end{teorema}
\begin{proof}[Dimostrazione]
    Dimostro per prima che la funzione $f$ è un morfismo di anelli, dove $f :  \mathbb{Z}_n \rightarrow \mathbb{Z}_{n_1} \times .. \times \mathbb{Z}_{n_k}$ è definita da $f(x) = (x \mod n_1, .., x \mod n_k)$ $\forall x \in Z$.
    \begin{itemize}
        \item $f(a + b) = ((a + b) \mod n_1, .., (a + b) \mod n_k) = \\ (a \mod n_1 + b \mod n_1, .., a \mod n_k + b \mod n_k) = \\
        (a \mod n_1, .., a \mod n_k) + (b \mod n_1, .., b \mod n_k) = \\
        f(a) + f(b)$ $\forall a, b \in \mathbb{Z}$.
        \item $f(1) = (1 \mod n_1, .., 1 \mod n_k)$ che non è altro che l'unità del prodotto diretto di anelli  $\mathbb{Z}_{n_1} \times .. \times \mathbb{Z}_{n_k}$.
        \item $f(ab) = ((ab) \mod n_1, .., (ab) \mod n_k) = \\ (a \mod n_1 \cdot b \mod n_1, .., a \mod n_k \cdot b \mod n_k) = \\
        (a \mod n_1, .., a \mod n_k) \cdot (b \mod n_1, .., b \mod n_k) = \\
        f(a) \cdot f(b)$ $\forall a, b \in \mathbb{Z}$.
    \end{itemize}

    Versione 1: \newline
    Siano x, y in $Z_n$ allora dimostro che se $[x] = [y]$ allora $f([x]) = f([y])$, ovvero che la funzione $f$ è ben definita.
    Dire $[x] = [y]$ equivale a dire che $n | (x - y)$. Dato che n è multiplo di $n_1, .. ,n_k$ per la proprietà transitiva segue che $(x - y)$ è a sua volta un multiplo di $n_1, .., n_k$. Quindi $[x] = [y]$ ma anche $[x]_{n1} = [y]_{n1}, ..., [x]_{nk} = [y]_{nk}$ e quindi $f$ è ben definita in quanto $f([x]) = f([y])$. \newline
    Dimostro ora che il morfismo $f$ è iniettivo, ovvero che il nucleo è banale. Sia $[a]_n \in ker(f)$, ovvero $f([a]_n) = [0]_n$. Ciò vuol dire che a è un multiplo comune a $n_1, n_2, .., n_k$. Poichè $MDC\{n_i, n_j\} = 1$ $\forall 1 \le i, j \le k$ e $i \neq j$ allora $mcm(n_1, ..., n_k) = n_1, .., n_k$ e quindi si può dire che $n_1, .., n_k / a$ che in termini di classi di equivalenza vuol dire $[a]_n = [0]_n$. \newline
    Inoltre dato che $\mathbb{Z}_n$ e $\mathbb{Z}_{n_1} \times .. \times \mathbb{Z}_{n_k}$ sono insiemi finiti e con la stessa cardinalità allora $f$ è necessariamente anche suriettiva. \newline
     Per il teorema di isomorfismo si ha che $\mathbb{Z}/{Ker(f)} \simeq \mathbb{Z}_{n_1} \times .. \times \mathbb{Z}_{n_k}$ come anelli ma anche che $Ker(f) = <n_1> \bigcap .. \bigcap <n_k> = <mcm\{n_1, .., n_k\}> = <n_1 \cdot .. \cdot n_k>$ poichè $n_i$ e $n_j$ sono coprimi $\forall i \neq j$. \newline
    Quindi $\mathbb{Z}/{Ker(f) = \mathbb{Z}/{<n>} = \mathbb{Z}_n}$ e l'isomorfismo $\Psi : \mathbb{Z}_n \rightarrow \mathbb{Z}_{n_1} \times .. \times \mathbb{Z}_{n_k}$ è quello dell'enunciato del teorema.
    \\
    
    Versione 2: \newline
     Dimostro poi che il morfismo $f$ è suriettivo ovvero che $Im(\Psi) = \mathbb{Z}_{n_1} \times .. \times \mathbb{Z}_{n_k}$. 
    Sia $(a_1 \mod n_1, .., a_k \mod n_k) \in \mathbb{Z}_{n_1} \times .. \times \mathbb{Z}_{n_k}$. \newline
    Dato che $MCD\{n_i, n_j\} = 1$ $\forall 1 \le i, j \le k$ com $i \neq j$. 
    Si ha la seguente identità di Bézeout:
    \begin{equation}
        c_in_i + b_in/n_i = 1
    \end{equation}
    ossia $u_i + v_i = 1$, dove $u_i := c_in_i \in <n_i>$ e $v_i := b_in/n_i \in <n/n_i>$. \newline
    Definisco $x:= a_1v_1 + .. + a_kv_k$ allora si ha che $f(x) = (a_1 \mod n_1, .., a_k \mod n_k)$ infatti:
    $$
    \begin{cases}
         v_i \mod n_j = 1 \mod n_j \iff i = j \\
         0 \ altrimenti. 
    \end{cases}
    $$
    Per il teorema di isomorfismo si ha che $\mathbb{Z}/{Ker(f)} \simeq \mathbb{Z}_{n_1} \times .. \times \mathbb{Z}_{n_k}$ come anelli ma anche che $Ker(f) = <n_1> \bigcap .. \bigcap <n_k> = <mcm\{n_1, .., n_k\}> = <n_1 \cdot .. \cdot n_k>$ poichè $n_i$ e $n_j$ sono coprimi $\forall i \neq j$. \newline
    Quindi $\mathbb{Z}/{Ker(f) = \mathbb{Z}/{<n>} = \mathbb{Z}_n}$ e l'isomorfismo $\Psi : \mathbb{Z}_n \rightarrow \mathbb{Z}_{n_1} \times .. \times \mathbb{Z}_{n_k}$ è quello dell'enunciato del teorema.
\end{proof}

\section{Logica modale}
\subsection{Sintassi della logica modale proposizionale e semantica di Kripke.}
\subsubsection*{Sintassi.}
La logica modale è un'estensione della logica proposizionale. L'alfabeto è quello della logica proposizionale a cui si aggiungono i connettivi modali, è costituito da:
\begin{itemize}
    \item insime numerabile di variabili (o formule atomiche).
    \item connettivi logici ($\neg$ negazione, $\land$ congiunzione e $\lor$ disgiunzione).
    \item simboli ausialiari (, ).
    \item connettivi modali: $\square$ scatola (o box) e $\diamond$ diamante.
\end{itemize}
Le parole del linguaggio sono le formule ben formate (FBF), definite in modo ricorsivo:
\begin{itemize}
    \item ogni variabile è una FBF.
    \item se A è una FBF, allora ($\neg A$), ($\square$A), ($\diamond A$) sono FBF.
    \item Se A e B sono FBF, allora ($A \land B$), ($A \lor B$), ($A \implies B$), ($A \iff B$) sono FBF.
\end{itemize}

Alcune letture dei simboli $\square$ e $\diamond$. \newline
La lettura più comune è $\square$A = "è necessario che A" e $\diamond$A = "è possibile che A". \newline
Secondo questa lettura i connettivi modali possono essere definiti uno nei termini dell'altro: 
\begin{itemize}
    \item $\square A \equiv \neg \diamond \neg A$, ovvero "è necessario A" è equivalente a "non è possibile non A". 
    \item $\diamond A \equiv \neg \square \neg A$, ovvero "è possibile A" è equivalente a "non è necessario non A".
\end{itemize}
Secondo logiche modali epistemiche $\square$A = "si sa che A". \newline
Secondo logiche modali deontiche $\square$A = "è obbligatorio che A". \newline
Secondo logiche modali doxastiche $\square$A = "si crede che A". \newline
Secondo logiche modali dimostrative $\square$A = "è dimostrabile che A".

\subsubsection*{Semantica dei mondi possibili o di Kripke.}
Come già visto, la logica proposizionale è una logica vero-funzionale: assegnando valori "0" o "1" alle variabili è possibile assegnare un valore complessivo "0" o "1" ad una formula in modo univoco applicando le regole della semantica. \newline
Per la logica modale la situazione è più complicata. Interpretando $\square$ come operatore di necessità e $\diamond$ come operatore di possibilità è possbile dire che le forumle $\square A \implies \diamond A$ e $A \implies \diamond A$ sono vere. Non si può dire lo stesso per $A \implies \square \diamond A$.

\begin{definizione}
    Un \textbf{frame} è una coppia (S, R), dove S è un insieme non vuoto, detto insieme dei mondi, e $\emptyset \ne R \subseteq S \times S$ è una relazione su S, detta relazione di accessibilità (se $(x, y) \in R$ si dice che y è accessibile da x). 
\end{definizione}
Un frame può essere rappresentato con un grafo diretto i cui vertici sono gli elementi dell'insieme S e ogni freccia va dal vertice x al vertice y se $(x, y) \in R$. 

\begin{definizione}
    Un \textbf{modello} su un frame (S, R) è una terna (S, R, V) dove $V \rightarrow \mathcal{P}(S)$ è la funzione di valutazione.
\end{definizione}
Una formula F si dice vera in un mondo $x \in S$ del modello $M = (S, R, V)$, e si scrive $M \models_x F$, se e solo se: 
\begin{itemize}
    \item F è una variabile: $M \models_x F$ significa che $x \in V(F)$.
    \item $F = \neg y$ e y è una variabile: $M \models_x F$ significa che $x \notin V(F)$.
    \item $F = \neg G$, dove G è una formula: $M \models_x F$ significa che $M \not\models_x G$.
    \item $F = G_1 \land G_2$: $M \models_x F$ significa che $M \models_x G_1$ e $M \models_x G_2$.
    \item $F = G_1 \lor G_2$: $M \models_x F$ significa che $M \models_x G_1$ o $M \models_x G_2$.
    \item $F = \square G$: $M \models_x F$ significa che $M \models_y G$ per ogni $y \in S$ tale che $(x, y) \in R$, ossia per ogni mondo y raggiungibile da x.
    \item $F = \diamond G$: $M \models_x F$ significa che $M \models_y G$ per qualche mondo $y \in S$ tale che $(x, y) \in R$, ossia per almeno un mondo raggiungibile da x.
\end{itemize}
Se come frame prendo $(\{\emptyset\}, R)$ dove $R \subseteq S \times S = \{(\emptyset, \emptyset)\}$, una funzione di valutazione è $V : var \rightarrow \mathcal{P}(S) = \{ \emptyset \, \{\emptyset\}\} \simeq \{0, 1\} $. 
In questo modo i connettivi modali diventano superflui e si "ricade" nella logica proposizionale. \newline
La vertià di una formula del linguaggio della logica modale dipenderà quindi dal frame scelto, in particolare, dalla relazione R.

\begin{definizione}
    Una formula F si dice \textbf{vera in un modello M=(S, R, V)}, e si scrive $M \models F$, se $M \models_x F$ $\forall x \in S$, ossia se è vera in tutti i mondi di S.
\end{definizione}

\begin{definizione}
    Una formula F si dice \textbf{valida in un frame (S, R)}, e si scrive $(S, R) \models F$, se è vera in tutti i modelli costruiti su (S, R).
\end{definizione}

\begin{definizione}
    Una formula si dice \textbf{valida} se e solo se è valida su ogni frame, e si scrive $\models F$.
\end{definizione}

Lo \textbf{schema di formule} è una collezione di formule aventi tutte la stessa forma sintattica. Ad esemprio con lo schema $\square F \implies F$ si intendono tutte le formule di questa forma.
Le tautologie della logica proposizionale sono valide su ogni frame.

\subsubsection*{Corrispondenza ed esprimibilità}
Si dice che un frame (S, R) gode di una certa proprietà se ne gode la relazione R. In molti casi una proprietà di un frame equivale alla validità di uno schema di formule modali nei frame con quella proprietà.

\subsection{Teorema sulla esprimibilià della proprietà riflessiva di una relazione.}
\begin{teorema}
    Lo schema $\square A \implies A$ è valido in un frame (S, R) se e solo se R è riflessiva.
\end{teorema}
\begin{proof}[Dimostrazione]
    Se R non è riflessiva, allora esiste un mondo $y \in S$ tale che $(y, y) \notin R$. \newline
    Sia $Z:= V(X)$, $Z \subseteq S$ tale che $y \notin Z$ (cioè X è falsa nel mondo y). \newline
    Sia $Z \subseteq \{z \in S : (y, z) \in R\}$ cioè X sia vera in tutti i mondi accessibili da y. Allora, se $M = (S, R, V)$ allora $M \models_y \square X$ ma $M \not\models_y X$. \newline

    Assumendo che lo schema $\square A \implies A$ sia valido in $(S, R)$, dimostro che la relazione è riflessiva. \newline
    Sia X una variabile, $s \in S$ e $V(X) = \{w \in S : (s, w) \in R\}$. Quindi $M \models_s \square X$ e dalla validità dello schema segue $M \models_s X$  quindi $w \in V(X)$ per cui $(w, w) \in R$ (dal primo punto della semantica). \newline
    
    L'altra implicazione, R riflessiva allora $\square A \implies A$ valido è quasi immediata. Infatti, se R è riflessiva e $\square A$ è vera nel mondo v in un modello, allora A è vera in tutti i mondi accessibili da v, in particolare A è vera in v stesso, ossia $\square A \implies A$.  
\end{proof}
Questo esempio sottolinea come la verità di un linguaggio della logica modale dipende dal frame scelto, ed in particolare dalla relazione R.\newline
Ad esempio, nella logica della necessità la formula $\square X \implies X$ deve essere sempre vera in quanto ha senso dire che "è necessario che X, allora X".\newline
Al contrario, nella logica deontica la stessa formula di legge "è obbligatorio che X, allora X" non ha bisogno di essere sempre vera quindi la relazione scelta non sarà riflessiva.

\subsection{Teorema sulla esprimibilià della proprietà simmetrica di una relazione.}
\begin{teorema}
    Lo schema $A \implies \square \diamond A$ è valido in un frame (S, R) se e solo se R è simmetrica.
\end{teorema}
\begin{proof}[Dimostrazione]    
    Assumendo R simmetrica, ovvero $(x, y) \in R \implies (y, x) \in R$, dimostro che la formula logica è valida. \newline
    Sia $M \models_w A$ e $(w, v) \in R$. Dato che $\exists (v, w) \in R$ allora $M \models_v \diamond A$. Ma questa non è altro che la proprietà della relazione $\forall v \in S$ tale che $(w, v) \in R$, allora $M \models_w \square \diamond A$. \newline
    
    Assumendo che lo schema $A \implies \square \diamond A$ sia valido in $(S, R)$, dimostro che la relazione è simmetrica. \newline
    Sia X una variabile e $V(X) = \{s\}$ e sia $t \in S$ tale che $(s, t) \in R$. 
    Quindi $M \models_s X$ e dalla valità dello schema segue che $M \models_s \square \diamond X$ da cui $M \models_t \diamond X$, Quindi esiste $r \in S$ tale che $(t, r) \in R$ e $M \models_r X$, ossia $r = s$.
\end{proof}

\subsection{Teorema sulla esprimibilià della proprietà transitiva di una relazione.}
\begin{teorema}
    Lo schema $\square A \implies \square \square A$ è valido in un frame (S, R) se e solo se R è transitiva.
\end{teorema} 
\begin{proof}[Dimostrazione]
    Assumendo R transitiva, ovvero $(x, y) \in R, (y, z) \in R \implies (x, z) \in R$, dimostro che la formula logica è valida. \newline 
    Sia $M \models_x \square A$, ossia $M \models_y A$ $\forall y \in S$ tale che $(x, y) \in R$.
    Sia $z \in S$ tale che $(y, z) \in R$. Allora per la transitività $(x, z) \in R$ e quindi $M \models_z A$, $M \models_y \square$  ossia $M \models_x \square \square A$ (risalendo nella catena delle relazioni). \newline

    Assumendo che lo schema $\square A \implies \square \square A$ sia valido su un frame $(S, R)$, dimostro che la relazione è transitiva. \newline
    Sia X una variabile, $s \in S$ e $V(X) = \{w \in S : (s, w) \in R\}$. Allora $M \models_s \square X$ e quindi, per la validità dello schema, $M \models_s \square \square X$, da cui $M \models_t \square X$ $\forall t \in S$ tale che $(s, t) \in R$, ossia $M \models_r X$ $\forall r \in S$ tale che $(t, r) \in R$, $(s, t) \in R$. \newline
    Da ciò segue che $r \in V(X)$ ossia $(s, t) \in R$ e $(t, r) \in R \implies (s, r) \in R$.
\end{proof}
\newpage

\section{Esercizi}
\subsection*{Campi finiti}
\begin{esercizio}
    Determinare se $f : M_{2\times 2} \rightarrow M_{2\times 2}$ è un morfismo di gruppi e/o anelli sapendo che data $M \in M_{2\times 2} \mapsto A \cdot M$.
\end{esercizio}
Dato che la moltiplicazione righe per colonne di due matrici $2 \times 2$ è ancora una matrice $2 \times 2$ allora $f$ è un morfismo di gruppi. \\
In particolare è un morfismo di gruppi abeliani se la matrice $A$ è l'identità in quanto il prodotto matriciale (righe $\times$ colonne) è commutativo solo in quel caso. \\
Non è un morfismo di anelli in quanto l'elemento $f(1) \neq I$.

\begin{esercizio}
    Scrivere tutti i sottogruppi di $\mathbb{Z}_4 = \{[0], [1], [2], [3]\}$.
\end{esercizio}
I sottogruppi di $\mathbb{Z}_4$ possono avere cardinalità 1, 2 o 4 (in quando la cardinalità del sottogruppo deve dividere quella del gruppo se questo è finito).
\begin{itemize}
    \item $<0> = \{[0]\}$ 
    \item $<1> = \mathbb{Z}_4$
    \item $<2> = \{[0], [2]\}$
    \item $<3> = \mathbb{Z}_4$
\end{itemize}
Dall'esercizio si evince anche la proprietà secondo cui l'insieme $<\overline{m}>$ genera $\mathbb{Z}_n$ se e solo se $n$ e $m$ sono coprimi.

\begin{esercizio}
    Indicare quali e quanti sono i sottogruppi di $\mathbb{Z}_{899}$.
\end{esercizio}
I sottogruppi di $\mathbb{Z}_{899}$ sono 4 dato che $899 = 31 \cdot 29$. Quindi c'è il sottogruppo banale di cardinalità 1, il sottogruppo $<31>$ di cardinalità 31, il sottogruppo $<29>$ di cardinalità 29 e infine l'altro sottogruppo banale $\mathbb{Z}_{899}$.

\begin{esercizio}
    Sia $G$ un gruppo con identità $e$. Sia $H$ un sottogruppo. Dire chi è $[e]$ in $G/H$.
\end{esercizio} 
$[e] = \{ g \in G : g \sim e\} = \{ g \in G : g = e \cdot h, h \in H\} = \{ g \in G : g = h, h \in H \} = H$. Quindi $[e] = H$.
\begin{esercizio}
        Siano $G_1$ e $G_2$ due gruppi abeliani e $H_1 \subseteq G_1$, $H_2 \subseteq G_2$ due sottogruppi. Dimostrare che, come gruppi, $(G_1 \times G_2)/(H_1 \times H_2) \simeq (G_1/H_1) \times (G_2/H_2)$.
\end{esercizio}
Sia $f : G_1 \times G_2 \rightarrow (G_1/H_1) \times (G_2/H_2)$ una funzione definita come $f(g_1, g_2) = ([g_1], [g_2])$ $\forall (g_1, g_2) \in G_1 \times G_2$. \\
Questa funzione è un morfismo di gruppi infatti: $f((g_1, g_2) + (k_1, k_2)) = f(g_1+k_1, g_2+k_2) = ([g_1+k_1], [g_2+k_2]) = f(g_1, g_2) + f(k_1, k_2)$ $\forall (g_1, g_2), (k_1, k_2) \in G_1 \times G_2$. Il morfismo è chiaramente suriettivo. \\
Inoltre $Ker(f)=\{(g_1, g_2) : f(g_1, g_2) = ([e_1, e_2])\} = H_1 \times H_2$ essendo $e_1$ e $e_2$ le identità di $G_1$ e $G_2$ rispettivamente. Quindi concludiamo usando il teorema di isomorfismo per gruppi abeliani (teorema \ref{teo:iso}).

\begin{esercizio}
    Sia G un gruppo ciclico finito tale che $|G| = m$. Mostrare che esiste un isomorfismo di gruppi $G \simeq \mathbb{Z}_m$.
\end{esercizio}
Sia $g \in G$ tale che $G = <g>$. La funzione $f: \mathbb{Z} \rightarrow G$ definita da $f(n) = g^n$ $\forall n \in \mathbb{Z}$ è un morfismo suriettivo di gruppi (dimostrato al teorema \ref{teo:sott_cicl}). \\
Dal teorema di isomorfismo (teorema \ref{teo:iso}) segue che, come gruppi, $G \simeq \mathbb{Z}/Ker(f)$. Poichè $Ker(f)$ è un sottogruppo di $\mathbb(Z)_n$, allora $Ker(f) = k \mathbb{Z} = <k>$ per qualche $k \in \mathbb{N}$. Dunque $m = |G| = |\mathbb{Z}_k| = k$ e come gruppi $G \simeq \mathbb{Z}_m$ (in quanto k = m). 

\begin{esercizio}
    Trovare, se esiste, un numero intero $x \in \mathbb{Z}$ tale che
    $$
        \begin{cases}
            x \equiv 2 \mod 3 \\
            x \equiv 3 \mod 5 \\
            x \equiv 1 \mod 8 
        \end{cases}
    $$  
\end{esercizio}
Dal teorema cinese dei resti (teorema \ref{teo:TCR}) si può dire che il numero intero $x$ esiste in quanto, $MDC\{3, 5\} = 1$, $MDC\{3, 8\} = 1$ e $MDC\{5, 8\} = 1$.
Dalla dimostrazione del teorema si ha che una soluzione del sistema di congruenze è data da $x := 2v_1 + 3v_2 + v_3$ dove se si considera una soluzione $(x, y)$ delle seguenti equazioni diofantee:
\begin{itemize}
    \item $3x + 40y = 1$
    \item $5x + 24y = 1$
    \item $x + 15y = 1$
\end{itemize}
allora:
\begin{itemize}
    \item $v_1 = 40y = 40 \cdot 1$
    \item $v_2 = 24y = 24 \cdot -1$
    \item $v_3 = 15y = 15 \cdot 1$
\end{itemize}
Si ha che $x = 2 \dot 40 + 3 \cdot -24 + 15 = -7 \in \mathbb{Z}$. Dunque $\overline{x} = \overline{-7} = \overline{113} \in \mathbb{Z}_{120}$ è una soluzione del sistema.

\begin{esercizio}
    Trovare, se esiste, un numero intero $x \in \mathbb{Z}$ tale che
    $$
        \begin{cases}
            x \equiv 0 \mod 2 \\
            x \equiv 2 \mod 3 \\
            x \equiv 1 \mod 4 
        \end{cases}
    $$  
\end{esercizio}
La prima congruenza stabilisce che $x$ deve essere pari, l'ultima che x deve essere dispari e quindi non esiste soluzione.

\begin{esercizio}
    Sia $f : \mathbb{Z} \rightarrow \mathbb{Z}_2 \times \mathbb{Z}_3 \times \mathbb{Z}_4$ il morfismo di anelli definito da $f(x) = (x \mod 2, x \mod 3, x \mod 4)$ $\forall x \in \mathbb{Z}$. Dire chi sono $Ker(f)$ e $Im(f)$.
\end{esercizio}
Si ha che $Ker(f) = <2> \cap <3> \cap <4> = <12> = <mcm\{3, 4\}>$. Dal teorema di isomorfismo per gruppi abeliani (teorema \ref{teo:iso}) si ha che $Im(f) \simeq \mathbb(Z)/<12> = \mathbb{Z}_{12}$. Quindi $|Im(f)| = 12$ e $Im(f) = \{f(0), f(1), .., f(11)\} = \{(0, 0, 0), (1, 1, 1), (0, 2, 2), (1, 0, 3),$ $(0, 1, 0), (1, 2, 1), (0, 0, 2), (1, 1, 3), (0, 2, 0), (1, 0, 1), (0, 1, 2), (1, 2, 3)\}$.

\begin{esercizio}
    Siano $m, n \in \mathbb{N}\backslash\{0,1\}$ e sia $f : \mathbb{Z} \rightarrow \mathbb{Z}_m \times \mathbb{Z}_n$ il morfismo di anelli definito da $f(x) = (x \mod m, x \mod n)$ $\forall x \in \mathbb{Z}$. Dire chi sono $Ker{f}$ e $Im(f)$.
\end{esercizio}
Si ha che $Ker(f) = <m> \cap <n> = <mcm\{m, n\}>$ e $Im(f) = \{f(0), .., f(n\cdot m - 1)\} \simeq \mathbb{Z}_{mcm\{m, n\}}$. 

\begin{esercizio}
    Mostrare che i gruppi $\mathbb{Z}_{144}$ e $\mathbb{Z}_6 \times \mathbb{Z}_{24}$ non sono isomorfi (e quindi non sono isomorfi neanche come anelli).
\end{esercizio}
Dato che $\mathbb{Z}_{144}$ è ciclico allora basta dimostrare che $\mathbb{Z}_6 \times \mathbb{Z}_{24}$ non è ciclico. Dato il teorema secondo cui ogni sottogruppo di un gruppo ciclico è ciclico (teorema \ref{teo:sott_cicl}), basta dire che il sottogruppo $G \subseteq \mathbb{Z}_6 \times \mathbb{Z}_{24}$ definito da $G :=\{(0, 0), (3, 0), (0, 12), (3, 12)\} \simeq \mathbb{Z}_2 \times \mathbb{Z}_2$ non è ciclico.

\subsection*{Logica modale}
\begin{esercizio}
    Dimostrare che lo schema di formule $\square (A \implies B) \implies (\square A \implies \square B)$ è valido su ogni frame.    
\end{esercizio}
Sia $w \in S$ un mondo e $M$ un modello su un frame $(S, R)$. Sia $M \models_w \square(A \implies B)$ e $M \models_w \square A$. \\
Basta controllare che quando $\square (A \implies B)$ è vera, allora se è vera $\square A$ allora è vera $\square B$. \\
Quindi $M \models_w \square(A \implies B)$ significa che $M \models_v A \implies B$ $\forall v \in S$ tale che $(w, v) \in R$, e analogamente $M \models_w \square A$ significa che $M \models_v A$ $\forall v \in S$ tali che $(w, v) \in R$. \\
Dunque $M \models_v B$ $\forall v \in S$ tali che $(w, v) \in R$, ossia $M \models_w \square B$. Allora lo schema di formule è dimostrato.

\begin{esercizio}
    Dimostrare che lo schema di formule $\square (A \land B) \iff (\square A \land \square B)$ è valido su ogni frame.
\end{esercizio}
Sia $w \in S$ un mondo e $M$ un modello su un frame $(S, R)$. \\
Sia $M \models_w \square(A \land B)$. Allora $M \models_v (A \land B)$ $\forall v \in S$ tale che $(w, v) \in R$ ovvero che $M \models_v A$ e $M \models_v B$.
Allora $M \models_w (\square A \land \square B)$, ovvero $M \models_w \square A$ e $M \models_w \square B$. \\
Dunque lo schema di formule è valido.

\begin{esercizio}
    Dimostrare che lo schema di formule $\diamond (A \lor B) \iff (\diamond A \lor \diamond B)$ è valido su ogni frame.
\end{esercizio}
Sia $w \in S$ un mondo e $M$ un modello su un frame $(S, R)$. \\
Sia $M \models_w \diamond (A \lor B)$. Allora  $M \models_v (A \lor B)$ se $\exists v \in S$ tale che $(w, v) \in R$, ovvero  $M \models_v A$ oppure  $M \models_v B$. 
Allora $M \models_w \diamond A$ oppure $M \models_w \diamond B$ ovvero $M \models_w \diamond A \lor \diamond B)$. \\
Lo schema di formule è valido.

\begin{esercizio}
     Dimostrare che lo schema di formule $\square (A \implies B) \implies (\diamond A \implies \diamond B)$ è valido su ogni frame.
\end{esercizio}
Sia $w \in S$ un mondo e $M$ un modello su un frame $(S, R)$. \\
Sia $M \models_w \square (A \implies B)$. Allora $M \models_v (A \implies B)$ $\forall v \in S$ tale che $(w, v) \in R$, ovvero $M \models_v A$ allora $M \models_v B$ per qualche $ v \in S$ tale che $(w, v) \in R$. Allora si può anche dire che $M \models_w \diamond B$ (il conseguente è sempre vero e quindi non c'è bisogno di controllare la verità dell'antecedente). \\
Lo schema di formule è valido.

\begin{esercizio}
     Dimostrare che lo schema di formule $\diamond (A \implies B) \implies (\square A \implies \diamond B)$ è valido su ogni frame.
\end{esercizio}
Sia $w \in S$ un mondo e $M$ un modello su un frame $(S, R)$. \\
Sia $M \models_w \diamond (A \implies B)$. Allora $M \models_v (A \implies B)$ se $\exists v \in S$ tale che $(w, v) \in R$, ovvero se $M \models_v A$ allora $M \models_v B$. Allora si può anche dire che $M \models_w \diamond B$ per qualche $v \in S$ tale che $(w, v) \in R$. \\
Lo schema di formule è valido.

\begin{esercizio}
    Dimostrare che lo schema di formule $\diamond A \implies \square A$ non è valido.
\end{esercizio}
Sia R la relazione definita da $R := \{(1, 1), (1, 2)\}$ e $V(A)=\{2\}$ (ovvero la formula A è valida solo nel mondo 2). \\
Allora $M \models_1 \diamond A$ e $M \not\models_1 \square A$, ossia $M \not\models_1 (\diamond A \implies \square A)$. \\
Lo schema di formule non è valido.

\begin{esercizio}
    Dimostrare che lo schema di formule $\square (A \implies B) \implies (\square A \implies \diamond B)$ non è valido.
\end{esercizio}
Sia R la relazione definita da $R := \{(1, 2)\}$ e $V(A) = V(B) = 1$. \\
Allora $M \models_2 \square (A \implies B)$ se e solo se $M \models_1 A \implies B$, che è vero. Allora $M \models_2 \square A$ in quanto $M \models_1 A$. Ma $M \not\models_2 \diamond B$ perchè non esiste $v \in \{1, 2\}$ tale che $M \models_v B$ e $(2, v) \in R$. \\
Lo schema di formule non è valido.

\begin{esercizio}
    Dimostrare che lo schema di formule $\square (\square A \implies B) \lor \square (\square B \implies A)$ non è valido.
\end{esercizio}
Sia $A = B = X$, con $X$ variabile, e la relazione $R := \{(1, 2), (2, 3)\}$ e $V\{X\} = 3$. \\ 
Allora $M \models_1 \square(\square X \implies X)$ se e solo se $M \models_2 \square X \implies X$ se e solo se $M \models_3 X$ e $M \models_2 X$, ma $M \not\models_2 X$. Quindi $M \not\models_1 \square(\square X \implies X)$.

\begin{esercizio}
    Dimostrare che lo schema di formule $\square (A \lor B) \implies (\square A \lor \square B)$ non è valido.
\end{esercizio}
Sia la relazione $R := \{(1, 1), (1, 2)\}$, $V\{A\} =  \{1\}$ e $V\{B\} =  \{2\}$. \\
Allora $M \models_1 \square (A \lor B)$ se e solo se $M \models_1 A \lor B$ e $M \models_2 A \lor B$, che è vero. Ma $M \not\models_1 (\square A \lor \square B)$, ovvero $M \models_1 \square A$ oppure $M \models_1 \square B$ che sono entrambi falsi. \\
La formula non è valida.

\begin{esercizio}
    Dimostrare che lo schema di formule $\square (\square A \implies A) \implies \square A$ non è valido.
\end{esercizio}
Sia la relazione $R := \{(1, 1), (1, 2), (2, 3)\}$ e $V\{A\} =  \{2, 3\}$. \\ 
Allora $M \models_1 \square (\square A \implies A)$ se e solo se $M \models_2 \square A \implies A$ se e solo se $M \models_3 A$ e $M \models_2 A$. Ma $M \not\models_1 \square A$ in quando $M \not\models A$.
\end{document}
